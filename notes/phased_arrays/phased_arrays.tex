% !TeX spellcheck = en_US
\documentclass[]{report}
\usepackage[utf8]{inputenc}
\usepackage{lmodern}
\usepackage{graphicx}
\usepackage{hyperref}
\usepackage{tikz}
\usepackage{amsmath}
\usepackage{amssymb}

\begin{document}

\section{Array Antennas}
% TODO: a lot of references to Balanis
% TODO: equation
- radiation characteristic for array antenna can be determined from rad characteristic of a singular element and the array factor
\begin{equation}
	S_{Array}(\theta, \phi) = S_{Element}(\theta, \phi) \cdot AF(\theta, \phi)
\end{equation}

- array factor is determined by treating every source of radiation in the array as in isotropic radiator and adding them (superposition for fields holds in linear media, e.g. free space)

- example: uniform linear array of N elements
\begin{equation}
	AF(\theta) = \sum_{n=0}^{N-1}\exp(jnkd\cos\theta)
\end{equation}

- k = wavenumber (explain), d = inter-element distance, no dependence along phi as its only 1D
% TODO: add some plots

- inter-element distance is chosen as smaller than wavelength lambda to avoid grating lobes, but has to be large enough to avoid physical overlap and near-field mutual coupling

- from plot, we see that array increases directivity strongly, but high directivity also requires precise alignment between transmitter and receiver

- if moving targets, precise alignment by mechanically orientating antennas bothersome, therefore techniques for beam-forming and beam-steering

\section{Beamforming and Beamsteering}

- explain beam-forming

- e.g. beam-steering is done for a phased array with a progressive phase

- upper element has to be this earlier in time
\begin{equation}
	t = \frac{s}{v_{ph}}
\end{equation}

- this equals a phase shift of
\begin{equation}
	\Delta \phi = k t v_{ph} = k s = k d\sin\alpha
\end{equation}

- illustration for the process in one dimension

- result for array factor in one dimension as plot

- this can be extended to two extensions and NxM elements
\begin{equation}
	\phi(m,n) = k\sin\theta(md_x\cos\phi + nd_y\sin\phi)
\end{equation}
- dx, dy refer to inter-element distances in their respective axes

- plot of result for two dimensions

- add profile for focusing the beam (e.g. bessel-beam, Gaussian beam?)

- how to achieve the phase shifts?

\section{Implementation}

- in the following:

- before radiation is emitted, electrical phase shifting

- after radiation is emitted, optics (e.g. lenses / reflectors)

- combination of both

\subsection{Electrical}

\subsection{Optical}

- lenses, change optical path length by refractive index; quite constant across frequency (non-dispersive), therefore don't limit bandwidth, but introduce losses, also consider matching (easier with silicon lenses) and therefore reflections
\begin{equation}
	\phi(l) = knl
\end{equation}
- refractive index, length through medium of refractive index

- describe the losses

- extended hemispherical lens (shape of a bullet) puts antenna at focus, cause almost planar wavefronts and therefore high directivity

- do the same for a parabolic reflector

\subsection{Combined}

- include this?

\end{document}          
