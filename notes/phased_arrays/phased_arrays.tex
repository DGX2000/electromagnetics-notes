% !TeX spellcheck = en_US
\documentclass[]{report}
\usepackage[utf8]{inputenc}
\usepackage{lmodern}
\usepackage{graphicx}
\usepackage{hyperref}
\usepackage{tikz}
\usepackage{amsmath}
\usepackage{amssymb}

\begin{document}

\section{Array Antennas}
% TODO: a lot of references to Balanis
The radiation characteristic of a complete array antenna is determined from the radiation characteristic of a singular element and the array factor:
\begin{equation}
	S_{Array}(\theta, \phi) = S_{Element}(\theta, \phi) \cdot AF(\theta, \phi) \text{ .}
\end{equation}
The array factor can be determined by treating every source of radiation in the array as an isotropic radiator and summing their effects as the principle of superposition holds in linear media (e.g. free space).

As an example, the array factor for an uniform linear array antenna of N elements is determined to be 
\begin{equation}
	AF(\theta) = \sum_{n=0}^{N-1}\exp(inkd\cos\theta)
\end{equation}
where \( k \) refers to the wavenumber (spatial frequency of the wave) and \( d \) refers to the inter-element distance. Wavenumbers can contain an imaginary part to model attenuation, but this won't be considered here. The inter-element distance is chosen to be smaller than the wavelength \( \lambda \) to avoid grating lobes [REF] but great enough to avoid physical overlap and mutual coupling in the near-field.
% TODO: Ref to balanis
% TODO: Add a plot

From the plot [REF, ADD PLOT] it can be seen that an array antenna is useful for achieving an increase in directivity as at point [...] all individual antennas constructively interfere. Highly directive antennas, however, require precise alignment between the transmitter and receiver. Furthermore, if either the transmitter or the receiver is moving, alignment by mechanically orientating antennas becomes too slow and bothersome, therefore techniques for forming and/or steering the beams through other means are interesting.

\section{Beamforming and Beamsteering}

The term beamforming will be used to denote the overarching category of techniques that dynamically change the radiation characteristic of an antenna. In this category the technique of beamsteering is very prominent. Beamsteering has the goal of concentrating the maximum directivity of the antenna towards a chosen target.


- upper element has to be this earlier in time
\begin{equation}
	t = \frac{s}{v_{ph}}
\end{equation}

- this equals a phase shift of
\begin{equation}
	\Delta \phi = k t v_{ph} = k s = k d\sin\alpha
\end{equation}

- illustration for the process in one dimension

- result for array factor in one dimension as plot

- this can be extended to two extensions and NxM elements
\begin{equation}
	\phi(m,n) = k\sin\theta(md_x\cos\phi + nd_y\sin\phi)
\end{equation}
- dx, dy refer to inter-element distances in their respective axes

- plot of result for two dimensions

- add profile for focusing the beam (e.g. bessel-beam, Gaussian beam?)

- how to achieve the phase shifts?

\section{Implementation}

- in the following:

- before radiation is emitted, electrical phase shifting

- after radiation is emitted, optics (e.g. lenses / reflectors)

- combination of both

\subsection{Electrical}

\subsection{Optical}

- lenses, change optical path length by refractive index; quite constant across frequency (non-dispersive), therefore don't limit bandwidth, but introduce losses, also consider matching (easier with silicon lenses) and therefore reflections
\begin{equation}
	\phi(l) = knl
\end{equation}
- refractive index, length through medium of refractive index

- describe the losses

- extended hemispherical lens (shape of a bullet) puts antenna at focus, cause almost planar wavefronts and therefore high directivity

- do the same for a parabolic reflector

\subsection{Combined}

- include this?

\end{document}          
