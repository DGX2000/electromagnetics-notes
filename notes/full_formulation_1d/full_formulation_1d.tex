% !TeX spellcheck = en_US
\documentclass[]{report}
\usepackage[utf8]{inputenc}
\usepackage{lmodern}
\usepackage{graphicx}
\usepackage{hyperref}
\usepackage{tikz}
\usepackage{amsmath}
\usepackage{amssymb}

\begin{document}

\section{Full FDTD for the 1D Case}
% TODO: Reference to that
- starting from six equations from maxwell from other note on plane waves
\begin{align}
	\partial_y E_z - \partial_z E_y &= -\mu \partial_t H_x \\
	\partial_z E_x - \partial_x E_z &= -\mu \partial_t H_y \\
	\partial_x E_y - \partial_y E_x &= -\mu \partial_t H_z \\
	\partial_y H_z - \partial_z H_y &= \varepsilon \partial_t E_x + \sigma E_x \\
	\partial_z H_x - \partial_x H_z &= \varepsilon \partial_t E_y + \sigma E_y \\
	\partial_x H_y - \partial_y H_x &= \varepsilon \partial_t E_z + \sigma E_z
\end{align}
- partial x refers to partial derivation

- only consider discretization along one axis, here z-axis so any derivative along another axis
can be ignored

- this gives the following six equations
\begin{align}
	\partial_z E_y &= \mu \partial_t H_x \\
	\partial_z E_x &= -\mu \partial_t H_y \\
	0 &= -\mu \partial_t H_z \\
	- \partial_z H_y &= \varepsilon \partial_t E_x + \sigma E_x \\
	\partial_z H_x  &= \varepsilon \partial_t E_y + \sigma E_y \\
	0 &= \varepsilon \partial_t E_z + \sigma E_z
\end{align}

% TODO: Add mention of Yee grid to plane wave note
- again discretize spatially and temporally in a staggered manner (Yee-grid)
- we can drop eqn 3. and 6, as they just state that there is no change
% TODO: Write down equations for that
\begin{align}
	\frac{E_y^{n+1/2}(k+1)-E_y^{n+1/2}(k)}{\Delta z} &= \mu \frac{H_x^{n+1}(k+1/2)-H_x^{n}(k+1/2)}{\Delta t} \\
	\frac{E_x^{n+1/2}(k+1)-E_x^{n+1/2}(k)}{\Delta z} &= -\mu \frac{H_y^{n+1}(k+1/2)-H_y^{n}(k+1/2)}{\Delta t} \\
	-\frac{H_y^{n}(k+1/2)-H_y^{n}(k-1/2)}{\Delta z} &= \varepsilon \frac{E_x^{n+1/2}(k)-E_x^{n-1/2}(k)}{\Delta t} + \sigma E_x^n(k) \\
	\frac{H_x^{n}(k+1/2)-H_x^{n}(k-1/2)}{\Delta z} &= \varepsilon \frac{E_y^{n+1/2}(k)-E_y^{n-1/2}(k)}{\Delta t} + \sigma E_y^n(k)
\end{align}

- new unknown variable, we can get that as simple average between two time steps, e.g.
\begin{equation}
	E_x^n(k) = \frac{E_x^{n+1/2}(k)-E_x^{n-1/2}(k)}{2}
\end{equation}

- no we get recurrence relations for the algorithm by multiplying out and then rearranging
\begin{align}
	H_x^{n+1}(k+1/2) &= H_x^{n}(k+1/2) + \frac{\Delta t}{\mu \Delta z}\left( E_y^{n+1/2}(k+1)-E_y^{n+1/2}(k) \right) \\
	H_y^{n+1}(k+1/2) &= H_y^{n}(k+1/2) - \frac{\Delta t}{\mu \Delta z}\left( E_x^{n+1/2}(k+1)-E_x^{n+1/2}(k) \right) \\
	E_x^{n+1/2}(k) &= E_x^{n-1/2}(k) - \frac{\Delta t}{\varepsilon \Delta z + \frac{1}{2}\sigma \Delta z \Delta t} \left( H_y^{n}(k+1/2)-H_y^{n}(k-1/2) \right) \\
	E_y^{n+1/2}(k) &= E_y^{n-1/2}(k) + \frac{\Delta t}{\varepsilon \Delta z + \frac{1}{2}\sigma \Delta z \Delta t} \left( H_x^{n}(k+1/2)-H_x^{n}(k-1/2) \right)
\end{align}
- in algorithm, first calculate 20/21, then 18/19

- again CFL condition for the time step
\begin{equation}
	\Delta t \leq \frac{\Delta z}{c}
\end{equation}

- for 18/19 term, we get
\begin{align}
	\frac{\Delta t}{\mu \Delta z} &= \frac{1}{Z_0 \mu_r} \\
	\frac{\Delta t}{\varepsilon \Delta z + \frac{1}{2}\sigma \Delta z \Delta t} &= \frac{\frac{\Delta z}{c}}{\varepsilon \Delta z + \frac{1}{2}\sigma \Delta z \frac{\Delta z}{c}} = \frac{Z_0}{\varepsilon_r + \frac{1}{2} Z_0 \sigma \Delta z}
\end{align}
% TODO: equation


\section{Simulations}


\end{document}          
