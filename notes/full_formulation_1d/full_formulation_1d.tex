% !TeX spellcheck = en_US
\documentclass[]{report}
\usepackage[utf8]{inputenc}
\usepackage{lmodern}
\usepackage{graphicx}
\usepackage{hyperref}
\usepackage{tikz}
\usepackage{amsmath}
\usepackage{amssymb}

\begin{document}

\section{Deriving FDTD method for one dimension}
After unfurling the curl operations we get to the following system of six equations
\begin{align}
	\partial_y E_z - \partial_z E_y &= -\mu \partial_t H_x \\
	\partial_z E_x - \partial_x E_z &= -\mu \partial_t H_y \\
	\partial_x E_y - \partial_y E_x &= -\mu \partial_t H_z \\
	\partial_y H_z - \partial_z H_y &= \varepsilon \partial_t E_x + \sigma E_x \\
	\partial_z H_x - \partial_x H_z &= \varepsilon \partial_t E_y + \sigma E_y \\
	\partial_x H_y - \partial_y H_x &= \varepsilon \partial_t E_z + \sigma E_z
\end{align}
that will function as basis for any of the following discretizations. Note that \(\partial_x\) is a shorthand for the partial derivation operator \(\frac{\partial}{\partial x}\).

\section{Simulations}


\end{document}          
