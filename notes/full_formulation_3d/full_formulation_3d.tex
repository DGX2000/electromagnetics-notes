% !TeX spellcheck = en_US
\documentclass[]{report}
\usepackage[margin=1in]{geometry}	% less margin to fit the longer formulas
\usepackage[utf8]{inputenc}
\usepackage{lmodern}
\usepackage{graphicx}
\usepackage{hyperref}
\usepackage{tikz}
\usepackage{amsmath}
\usepackage{amssymb}

\begin{document}

\section{Full FDTD for the 3D Case}
- again start with maxwells

- currents \( J_x, J_y, J_z \) were added as they are necessary for voltage and current sources

- also anisotropy is possible as permittivity, permeability and conductivity no can vary depending on direction

\begin{align}
	\partial_y E_z - \partial_z E_y &= -\mu_x \partial_t H_x \\
	\partial_z E_x - \partial_x E_z &= -\mu_y \partial_t H_y \\
	\partial_x E_y - \partial_y E_x &= -\mu_z \partial_t H_z \\
	\partial_y H_z - \partial_z H_y &= \varepsilon_x \partial_t E_x + \sigma_x E_x + J_x \\
	\partial_z H_x - \partial_x H_z &= \varepsilon_y \partial_t E_y + \sigma_y E_y + J_y \\
	\partial_x H_y - \partial_y H_x &= \varepsilon_z \partial_t E_z + \sigma_z E_z + J_z \text{ .}
\end{align}
The operator \( \partial_x \) is a shorthand for the partial derivative \( \frac{\partial}{\partial x} \).

- no terms are going to be dropped now

- since form of eqn 1-3 and 4-6 are similar, well only do 1 and 4 and then transfer those results

- the grid will look like this now (add grid and coordinate system indication = left-handed system here)

- node is indicated by a 3-tuple (i,j,k) where x..., y..., z...

- to make the following expressions shorter i,j,k (current node) are taken as default parameters, only the relevant differing parameters are written out, e.g. i+1, j+1, or k+1

- E(i,j,k) becomes E, E(i,j+1,k) would become E(j+1)

- now for equation 1, after discretization
\begin{equation}
	\frac{E_z^{n+1/2}(j+1) - E_z^{n+1/2}}{\Delta y} - \frac{E_y^{n+1/2}(k+1) - E_y^{n+1/2}}{\Delta z} = -\mu_x \frac{H_x^{n+1}-H_x^n}{\Delta t}
\end{equation}

- rearranging to the desired ...

\begin{equation}
	H_z^{n+1} = H_z^n - \frac{\Delta t}{\mu_x \Delta y}\left( E_z^{n+1/2}(j+1) - E_z^{n+1/2}  \right) + \frac{\Delta t}{\mu_x \Delta z}\left( E_y^{n+1/2}(k+1) - E_y^{n+1/2}  \right)
\end{equation}

- in turn for equation 4, after discretization

\begin{equation}
	\frac{H_z^n(j+1)-H_z^n}{\Delta y} - \frac{H_y^n(k+1)-H_y^n}{\Delta z} = \varepsilon_x \frac{E_x^{n+1/2}-E_x^{n-1/2}}{\Delta t} + \sigma_x \frac{E_x^{n+1/2}-E_x^{n-1/2}}{2} + J_x
\end{equation}

- rearranging to ...

\begin{equation}
	E_x^{n+1/2} = E_x^{n+1/2} + \frac{2\Delta t}{2\varepsilon_x + \Delta t \sigma_x}\left( \frac{H_z^n(j+1) - H_z^n}{\Delta y} - \frac{H_y^n(k+1) - H_y^n}{\Delta z} - J_x \right)
\end{equation}

- write out the results for all 6 equations

- cfl condition
\begin{equation}
	\Delta t \leq \frac{1}{c\sqrt{\frac{1}{(\Delta x)^2} + \frac{1}{(\Delta y)^2} + \frac{1}{(\Delta z)^2}}}
\end{equation}

- cell size, depends on simulation wavelength (chosen frequency), as general high-frequency principle no dimension of cell size should be larger than one tenth of wavelength

- example 2.4ghz leads to 12.5cm wavelength

- therefore 1.25 cell size is ok

\begin{equation}
	\Delta t \leq = \frac{1}{c \sqrt{\frac{3}{(1.25cm)^2}}} = \frac{1.25cm}{\sqrt{3}c} \approx 24ps
\end{equation}

\section{Simulations}

- i won't write any matlab scripts for this one, would be a waste of effort

- instead, head to next note on building the grid

- from that point, focus should be spent on implementing a functioning toolchain

- toolchain consists of: parametric modeller, grid builder, simulator, result visualizer

- none of these need to be fancy, parametric modeller only needs to export some boxes for a start, grid builder needs problem configuration and putting in a source, e.g. voltage source, simulator should probably be in grid builder, result visualizer should provide a good way of judging vector fields (plane cuts and 3D view with many lines, thickness/color depends on magnitude or something)


\end{document}          
